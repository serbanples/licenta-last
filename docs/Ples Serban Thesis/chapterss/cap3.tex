
\chapter{Architecture and Technologies}

\section{Architecture}

The application uses a combination of Event Driven architecture and Request-Response architecture, making as much use as possible of Node JS's\cite{NODEJS} built in capabilities for handling multiple operations on a single thread using concurrency and asynchronous programming.

The application consists of multiple microservices. These microservices implement communication patterns like \textit{Request-Response} and \textit{Publish/Subscribe}. All microservice communication is done using message queues like RabbitMQ and BullMQ. While RabbitMQ offers great support for both Request-Response communication and Publish/Subscribe communication, I chose to use BullMQ for Publish/Subscribe. BullMQ uses Redis to queue messages and process them. By using Redis, BullMQ offers an easy way to schedule the times when a message is executed and implement retry policies. On the other hand, RabbitMQ is built upon AMQP (Advanced Message Queuing Protocol)\cite{AMQP}. This makes it possbile to pass messsages across an internal network and register handlers for these messages. 

Since this application is heavily focused on microservices, here is a short description for each one of them:
\begin{itemize}
    % \item Client - User interface, hosted on a server of its own.
    \item \textbf{Webserver Service} - Single HTTP server, tasked with serving the client application and providing HTTP routes for fetching data.
    \item \textbf{Authentication Service} - Service handling all authentication logic. This server is only accessible via RabbitMQ queues.
    \item \textbf{Core Service} - Service handling all reads and writes to the database. This server is accessbile via RabbitMQ queues.
    \item \textbf{Authorization Service} - Service handling authorization across the whole application. The application implements a RBAC (Role Based Access Control). Each request goes through this service before reaching the Core service. Only accessbile via RabbitMQ.
    \item \textbf{Uploader Service} - Service handling all file uploads. Since uploads are usually resource intensive tasks and can fail due to numerous reasons, this is a processor service using BullMQ.
    \item \textbf{Mail Service} - Service handling all emails sent. Sending emails is also a resource intensive task especially if sending to multiple users at the same tim. This is a processor service using BullMQ.
    \item \textbf{Websocket Server} - Server exposing only a websocket connection. This server is handling all of the chat features inside of the application, listening for notifications sent by clients, invoking the Core service for different writes (saves, updates) to the database, and dispatching a notification to the intended client.
    \item \textbf{Notification Server} - Server exposing a route for handling SSE (Server Sent Events). This server is the one handling in-app notifications. 
\end{itemize}

\section{Technology Stack}

The \emph{Cloud Class} application is a fully browser-based system built end-to-end in TypeScript/JavaScript.  Both client and server leverage modern, asynchronous architectures and microservices to ensure scalability, maintainability, and real-time interactivity.  This section outlines the key technologies and frameworks that form the backbone of my solution.

\subsection{Server Implementation}

The backend of \emph{Cloud Class} is implemented as a suite of modular microservices responsible for business logic, data persistence, and event handling.  All services are written in \mbox{TypeScript}, compiled to JavaScript, and orchestrated via Docker.

\begin{itemize}
  \item \textbf{TypeScript}  
    \begin{itemize}
      \item A statically typed superset of JavaScript that compiles to ES2019.  
      \item Enables interface-driven design, strong type checking, and advanced IDE support (e.g., IntelliSense).  
      \item Facilitates early detection of errors, consistent code formatting (TSLint / ESLint), and straightforward refactoring in large codebases.
    \end{itemize}

  \item \textbf{Node.js}  
    \begin{itemize}
      \item Event-driven, non-blocking I/O runtime built on Chrome’s V8 engine.  
      \item Ideal for handling high concurrency with minimal resource overhead.  
    \end{itemize}

  \item \textbf{NestJS}  
    \begin{itemize}
      \item Progressive Node.js framework leveraging decorators and dependency injection.  
      \item Promotes a modular architecture: controllers, providers, and gateways can be developed and tested independently.  
      \item Native support for REST, GraphQL, WebSockets, and microservice patterns (TCP, gRPC, Redis-based transport).
    \end{itemize}

  \item \textbf{MongoDB}  
    \begin{itemize}
      \item NoSQL document database storing data in flexible BSON format.  
      \item Supports horizontal scaling via sharding, automatic replica sets for high availability, and ACID transactions across multiple documents (v4.0+).  
      \item Optimized for rapid development cycles, enabling schema evolution without downtime.
    \end{itemize}

  \item \textbf{MinIO (S3-compatible)}  
    \begin{itemize}
      \item High-performance, Kubernetes-native object storage with Amazon S3 API compatibility.  
      \item Features erasure coding, bitrot protection, server-side encryption, and multi-tenant isolation.  
    \end{itemize}

  \item \textbf{Real-Time Communication}  
    \begin{itemize}
      \item \emph{Socket.IO}: Bidirectional WebSocket library with automatic fallback to HTTP long-polling. Used for live chat, collaborative editing, and presence notifications.  
      \item \emph{Server-Sent Events (SSE)}: Lightweight, one-way event streams for broadcasting state changes (e.g., file uploads progress, system alerts) with minimal overhead.
    \end{itemize}

  \item \textbf{Logging \& Analytics:}  
    \begin{itemize}
      \item \emph{Elasticsearch \& Kibana}: Distributed search and analytics engine powering full-text search, structured queries, and real-time log aggregation.  
      \item Logs are shipped via Filebeat into an ELK stack, enabling dashboards for performance monitoring, error tracking, and usage metrics.
    \end{itemize}

  \item \textbf{Testing}  
    \begin{itemize}
      \item \emph{Jest}: Zero-configuration JavaScript testing framework for unit and integration tests, featuring snapshot testing and code coverage reports.  
    \end{itemize}

  \item \textbf{Message Queues \& Background Jobs:}  
    \begin{itemize}
      \item \emph{RabbitMQ}: AMQP broker for reliable, complex routing of messages between microservices (e.g., task dispatch, order processing).  
      \item \emph{BullMQ}: Redis-backed job queue for scheduled and repeatable tasks such as email delivery, report generation, and notification bursts.  
      \item Combined to balance high-throughput event streams (RabbitMQ) with lightweight, retry-capable background jobs (BullMQ).
    \end{itemize}
\end{itemize}

\subsection{Client Implementation}

The frontend of \emph{Cloud Class} is a single-page application delivering responsive, component-driven user experiences.  Written in TypeScript and React, it communicates with backend microservices via REST, and real-time channels.

\begin{itemize}
  \item \textbf{HTML5 \& Semantic Markup}  
    \begin{itemize}
      \item Establishes the document structure with accessibility in mind (ARIA roles, landmarks).  
    \end{itemize}

  \item \textbf{CSS3 \& Responsive Design}  
    \begin{itemize}
      \item Utilizes Flexbox, Grid layouts, and custom properties (CSS variables) for adaptable, maintainable styling.  
    \end{itemize}

  \item \textbf{React \& TypeScript}  
    \begin{itemize}
      \item Component-based architecture with JSX syntax and a virtual DOM for efficient UI updates.  
    \end{itemize}

  \item \textbf{Real-Time Updates (Socket.IO Client)}  
    \begin{itemize}
      \item Establishes persistent WebSocket connections for instant notifications (e.g., new messages, collaborative whiteboard changes).  
      \item Automatic reconnection and event buffering ensure continuity during transient network issues.
    \end{itemize}
\end{itemize}


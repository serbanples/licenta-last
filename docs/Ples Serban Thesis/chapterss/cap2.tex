%%%%%%%%%%%%%%%%%%%%%%%%%%%%%%%%%%%%%%%%%%%%%%%%%
%%%%%%%%%%%% chap: title %%%%%%%%%%%%%%%%%
%%%%%%%%%%%%%%%%%%%%%%%%%%%%%%%%%%%%%%%%%%%%%%%%%

\chapter{Related Work}\label{chapter:chap2}



\section{Existing Solution}

After careful research, some similar applications would include the following:
\begin{enumerate}
    \item \textbf{Slack}\cite{Slack} - Slack is a widely used team collaboration tool, particularly popular in professional and academic contexts. It enables users to create channels for specific topics or projects, making it suitable for organizing discussions and sharing resources. Additionally, Slack supports integrations with various applications and offers video call functionality, making it a versatile platform. However, Slack is designed primarily for professional teams and its interface may not specifically address the unique needs of student collaboration and study organization.
    \item \textbf{Google Classroom}\cite{Classroom} - Google Classroom is widely used in educational settings for resource sharing and assignments. While it is more of an educational management tool, it integrates with Google Meet for video meetings and supports file sharing.
    \item \textbf{Microsoft Teams}\cite{Teams} - Teams is used extensively in academic and corporate settings for collaboration. It supports chats, file sharing, and video meetings, making it a good platform for group study environments.
    \item \textbf{NutriBiochem Mobile Application}\cite{NutriBiochem} - The NutriBiochem mobile application was developed as a learning resource for use in undergraduate biochemistry and nutrition education; specifically, the app was created for use in two courses in undergraduate Biochemistry and Nutrition at the University of Guelph Humber, although it was made freely available to the public through several different app stores and was promoted to students in several Biochemistry courses at the University of Guelph. While designed specifically to address the needs of these two specific courses, the content of the app (described below) is generalizable to most undergraduate courses that cover the fundamentals of human metabolism and nutrition. The goal of this project was to create a highly interactive, usable app that could be used as a mobile learning tool, and to study its use and pedagogical impact.
\end{enumerate}

\section{Gaps in Current Solution}

Scientific articles can be found in different online database:
\begin{enumerate}
    \item bubu
\end{enumerate}

\section{Proposed Solution}

For each read paper try to note information, so you can use later:
% \begin{itemize}
%     \item resource author;
%     \item resource name (article title, book title, ...);
%     \item subject: what is the paper aim;
%     \item addressed problem ; 
%     \item methods/methodology used to solve the problem;
%     \item used algorithms;
%     \item test data (e.g. benchmarks, real cases).
% \end{itemize}

% For a git resource or an existing application:
% \begin{itemize}
%     \item URL;
%     \item application features;
%     \item used technologies.
% \end{itemize}
\chapter{Related Work}\label{chapter:chap2}

\section{Existing Solutions}

A number of established platforms address parts of the collaboration and organization problem, but none combine document sharing, chat, study scheduling and real‐time video in a student‐centric package:

\paragraph{Slack}\cite{Slack}  
Slack is widely adopted in professional and academic circles for team communication.  Its channel-based model and rich integration ecosystem (e.g.\ file attachments, bots, external APIs) make it ideal for project-driven discussion.  However, Slack’s interface and feature set are geared toward corporate workflows; it lacks built-in study-session scheduling or academic resource management, and its cost structure can be prohibitive for student use.

\paragraph{Google Classroom}\cite{Classroom}  
Google Classroom streamlines assignment distribution, grading, and resource sharing within educational institutions.  It integrates seamlessly with Google Meet for video sessions and Google Drive for file storage.  While strong on assignment workflows, Classroom is not designed for open‐ended study groups or peer-to-peer chat outside of formal class rosters, and offers limited flexibility in customizing collaborative spaces.

\paragraph{Microsoft Teams}\cite{Teams}  
Microsoft Teams unifies chat, file sharing, and video conferencing under the Office 365 umbrella.  Its “Teams” and “Channels” structure supports both class-level and small group discussions, and includes a built-in calendar.  Despite its power, the learning curve and enterprise focus of Teams can overwhelm students, and it provides few features tailored specifically to student-driven study sessions.

\paragraph{NutriBiochem Mobile Application}\cite{NutriBiochem}  
The NutriBiochem app was crafted for undergraduate biochemistry and nutrition courses, offering interactive content and quizzes.  Its design demonstrates how a domain-specific tool can deeply engage learners in a particular subject area.  However, its tightly focused scope limits reuse across varied disciplines, and it omits real-time communication or scheduling capabilities common to collaborative study.

\section{Gaps in Current Solutions}

Although these platforms excel at discrete tasks—chat (Slack, Teams), course management (Google Classroom), or subject-focused learning (NutriBiochem)—several gaps remain:

First, none provide a unified, student-driven environment that seamlessly combines file management, ad hoc group chat, and on-demand study‐session scheduling. Students often juggle multiple services (e.g.\ Slack for chat, Google Drive for documents, Calendly for scheduling), which introduces friction and context-switching overhead.

Second, current tools emphasize instructor-led workflows or corporate use cases; they lack features like flexible “study rooms,” lightweight event RSVPs, and peer-to-peer resource tagging that empower students to take ownership of their collaborative learning.

Finally, integration with external calendars and real-time video is either missing or bolted on as a generic add-on, rather than designed around the rhythm of student study sessions and group learning.

\section{Proposed Solution}

In response to the limitations of existing tools, I have developed a web application focused on two core capabilities: file sharing and real-time chat.  By combining these into a single, student-centric platform, the goal is to reduce the fragmentation and context-switching that students experience when juggling separate services.

\begin{itemize}
  \item \textbf{File Upload \& Repository:}  
    Users can upload lecture notes, articles, and other learning materials directly into organized collections.  Each file is stored securely, versioned, and can be tagged or searched by course, topic, or keyword.  Peers in the same study group can browse and download shared resources without leaving the application.
  
  \item \textbf{Real-Time Chat Groups:}  
    Ad-hoc chat rooms allow students to form topic-specific discussion channels on the fly.  Messages are delivered instantly, with “seen” indicators and simple reaction support.  The chat interface is designed for quick group Q\&A, file previewing, and link sharing, all within the same context as the file repository.
\end{itemize}

The architecture is deliberately modular—each feature lives in its own microservice—so that future additions (for example, scheduling study sessions or integrating with external calendars) can be slotted in with minimal disruption.  
